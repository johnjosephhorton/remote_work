\documentclass[12pt]{article}

\usepackage{booktabs}
\usepackage{dcolumn} 
\usepackage{epstopdf}
\usepackage{graphicx}
\usepackage{hyperref}
\usepackage{longtable} 
\usepackage{natbib}
\usepackage{rotating}
\usepackage{tabularx}
\usepackage{amsmath}
\usepackage{setspace}
\usepackage{caption}
\usepackage{epigraph}

\usepackage[super]{nth}  
\hypersetup{
  colorlinks = TRUE,
  citecolor=blue,
  linkcolor=red,
  urlcolor=black
}

\hypersetup{colorlinks = TRUE, citecolor=blue, linkcolor=red, urlcolor=black}

\DeclareMathOperator*{\argmax}{arg\,max}

\newcommand{\starlanguage}{Significance indicators: $p \le 0.05:*$, $p \le 0.01:**$ and $p \le .001:***$.}

\newcommand{\covid}{COVID-19}  

\newtheorem{proposition}{Proposition}
\newtheorem{assumption}{Assumption}
\newtheorem{example}{Example}
\newtheorem{observation}{Observation}
\newtheorem{lemma}{Lemma}

\newcommand{\important}[1]{\textcolor{blue}{\textbf{ #1}}}
\newcommand{\quantclaim}[1]{\textcolor{red}{\textbf{ #1}}}


\newcommand{\LFPRhat}{56}
\newcommand{\numObs}{25,001}
\newcommand{\numObsWorking}{13,937}
\newcommand{\SurveyStart}{2020-05-02}
\newcommand{\SurveyEnd}{2020-05-08}
\newcommand{\LaidOff}{10.1}
\newcommand{\LaidOffLB}{8.6}
\newcommand{\LaidOffUB}{11.7}
\newcommand{\WFH}{35.2}
\newcommand{\WFHLB}{33.9}
\newcommand{\WFHUB}{36.6}
\newcommand{\alreadyWFH}{15.0}
\newcommand{\alreadyWFHLB}{13.5}
\newcommand{\alreadyWFHUB}{16.5}
\newcommand{\stillCommute}{37.1}
\newcommand{\stillCommuteLB}{35.8}
\newcommand{\stillCommuteUB}{38.4}

\begin{document} 

\title{COVID-19 and Remote Work:\\ An Early Look at US Data}

\date{Very Preliminary \\ \today}

\date{\today}

\author{Erik Brynjolfsson\\Stanford \& NBER \and John Horton\footnote{
    MIT's COUHES ruled this project as exempt (number E-2075).
    Code and Data here: \href{https://github.com/johnjosephhorton/remote_work/}{https://github.com/johnjosephhorton/remote\_work/}.
    If you have questions, please email us.
  }\\MIT \& NBER \and Adam Ozimek\\Upwork \and Daniel Rock\\MIT \and Garima Sharma\\MIT \and Hong Yi Tu Ye\\MIT}


\maketitle

\begin{abstract}
  \noindent 
  We report the results of a nationally representative sample of the US population on how they are adapting to the \covid{} pandemic.
  The survey was started on April 1, 2020 and is on-going.
  Of those previously employed, \LaidOff{}\% (95\% CI is [\LaidOffLB,\LaidOffUB]), report being laid-off or furloughed in the last 4 weeks;
  \WFH{}\% (95\% CI is [\WFHLB,\WFHUB]) report they were previously commuting and are now working from home.
  %The percentage already working from home pre-Covid-19 is \alreadyWFH{}\% (95\% CI is [\alreadyWFHLB,\alreadyWFHUB]). 
  We document substantial state-level geographic variation in responses.
  There is a strong negative relationship between the fraction in a state still commuting to work and the fraction working from home. 
  It seems probable that many workers currently commuting could be converted to remote workers. 
  However, using data on state UI claims, we find that states with higher fractions of workers still commuting have lower-than-expected UI claims; states with more newly remote workers have higher-than-expected UI claims.
%  The timing of singups is consistent with 
  \newline 
%% \noindent JEL J01, J24, J3 \newline 
%% keywords: bargaining, match formation, wage determination
\end{abstract} 


\onehalfspacing 

\section{Introduction}
  \noindent {\color{red} Preliminary! Please contact us before relying on these numbers.} \\
  \noindent {\color{red} Latest draft: \href{https://www.john-joseph-horton.com/papers/remote\_work.pdf}{https://www.john-joseph-horton.com/papers/remote\_work.pdf}} \newline
  \noindent {\color{red} Code \& Data (please use): \href{https://github.com/johnjosephhorton/remote\_work/}{https://github.com/johnjosephhorton/remote\_work/}} \newline
  {\color{red} \rule{\linewidth}{0.5mm}}

The on-going \covid{} pandemic has confined large numbers of people to their homes via quarantines and shelter-in-place orders.
Large numbers of businesses are closed. 
There have already been enormous and unprecedented increases in workers filing unemployment insurance claims \citep{goldsmith2020}. 
To get a real-time sense of how firms and workers are responding, we conducted a Google Consumer Surveys (GCS).\footnote{
GCS is a relatively low-cost tool for rapidly collecting responses to simple questions \cite{stephens2014hands}. 
GCS charges \$0.05 per response for a single question, substantially less than professional survey firms charge for more complicated products, and response representativeness is often comparable to similar alternatives \citep{santoso2016survey, brynjolfsson2019using}.
}

We asked a single question:
````Have you started to work from home in the last 4 weeks?''
with the following response options: 
\begin{enumerate} 
\item ``I continue to commute to work''
\item ``I have recently been furloughed or laid-off''
\item ``Used to commute, now work from home''   
\item ``Used to work from home and still do''       
\item ``Used to work from home, but now I commute''
\item ``None of the above / Not working for pay''
\end{enumerate} 

We launched our survey on \SurveyStart{}; this report includes responses up until \SurveyEnd{}. 
So far, we have surveyed a total of \numObs{} respondents.

\section{Results}

Of the respondents, \numObsWorking{} reported something other than ``None of the above...''
This gives an implied employment rate of \LFPRhat{}\%, which is slightly lower than the BLS estimate of about 60\%.\footnote{
  \url{https://fred.stlouisfed.org/series/EMRATIO}
}
We restrict our sample those reporting being employed 4 weeks prior.

The distribution of answers pooled over all respondents is shown in Figure~\ref{fig:working_summary}. 
We can see that the most common response from workers was that they continue to commute, at \stillCommute{}\% (95\% CI is [\stillCommuteLB,\stillCommuteUB]). 
But the next most common was that they were now working from home. 

The \emph{now} working from home fraction is about \WFH{}\%, suggestintg the \cite{dingel2020} estimate of 34\% is an underestimate, as we observe that a substantial fraction were already working from home: \alreadyWFH{}\% reporting they were already working from home pre-COVID-19.\footnote{


  

  
The ATUS release has 23.7\% working from home on an average day, though our question implies working from home all the time; the from-home-only fraction in the ATUS is 18.2\%. 
Our \WFH{}\% is also broadly consistent with the ``Freelancing in American Survey'' that reported 16.8\% report doing most of all of their work remote \citep{upwork2019}, though this included people working from co-working spaces, coffee shops, homes and so on.
The Census 2019 gives 5.3\% as ``working from home some'' so there is clearly lot of respondent uncertainty about what precisely various questions mean.
}
%The BLS reported fraction is TK. 
% As we will see, there is some regional variation, though we make no attempt to account for differing regions having different job compositions.
% upwork2019,

If we take the US labor force at about 165M, with an 11\% increase in furloughs/lay-offs, the implication is about 16M Americans are recently out of work.
The total UI filings for the last two weeks adds up to 9.9M.\footnote{
  \url{https://www.dol.gov/ui/data.pdf}
}
However, not all unemployed have yet filed for unemployment.
Given the decline in hiring, Prof. Wolfers estimates we are dealing with about 16M unemployed, which matching our point estimate.\footnote{
  \url{https://www.nytimes.com/2020/04/03/upshot/coronavirus-jobless-rate-great-depression.html}
}

% https://fred.stlouisfed.org/series/CLF16OV
%% This number makes total sense.  However, no everyone laid off applies for unemployment. And that data is only up through March 28. This number is very very plausible. In fact looking at all the UI data and also lower hiring rate, Wolfers estimates that we are dealing with 16 m so far, so you are like exactly on point 

\begin{figure}
  \caption{Answers to the question ``Have you started to work from home in the last 4 weeks?'', conditional upon being in the labor force from a US sample} \label{fig:working_summary}
\centering
\begin{minipage}{1.0 \linewidth}
  \includegraphics[width = \linewidth]{plots/working_summary.pdf} \\
  \begin{footnotesize}
    %% \begin{singlespace}
    %%   \emph{Notes:} 
    %% \end{singlespace}
    \end{footnotesize}
\end{minipage}
\end{figure} 

GCS also infers respondent gender.
We analyzed responses by gender but did not find any notable differences.
See Appendix~\ref{sec:gender} for this analysis. 

\subsection{Geographic variation} 
COVID-19 has affected various parts of the US differently, with the main epicenter in New York City.
In Figure~\ref{fig:region}, we plot the fraction of respondents choosing each answer, by region.
GCS captures a respondent's city and state, which are then mapped to the regions ``Northeast'', ``Midwest'', ``West'' and ``South.'' 

In the first facet from the left, we can see that the South has the highest fraction still commuting to work and the Northeast has the lowest. 
In the second facet from the right, we can see that the Northeast has the highest fraction of respondents switching to working from home, and the South the fewest.
The Northeast started from the lowest fraction working from home, though these fractions are imprecisely estimated and are all fairly similar to each other. 
The Northeast fraction now working from home is over 40\%. 

\begin{figure}
  \caption{Responses by US region} \label{fig:region}
\centering
\begin{minipage}{1.0 \linewidth}
  \includegraphics[width = \linewidth]{plots/region.pdf} \\
  \begin{footnotesize}
%    \begin{singlespace}
%      \emph{Notes:} Standard errors are reported. 
%    \end{singlespace}
    \end{footnotesize}
\end{minipage}
\end{figure} 

For a finer-grained look, we plot responses by state in Figure~\ref{fig:geo}.
It is important to keep in mind that some of these point estimates are fairly imprecise.

The fraction of workers that are still continuing to commute to work is highest in the Dakotas, Wyoming and Montana.
There is still a substantial fraction in the South continuing to commute. 
The Northeast, with the exception of Vermont, shows large reductions in people still commuting to work. 

\begin{figure}
  \caption{Responses by US State} \label{fig:geo}
\centering
\begin{minipage}{1.0 \linewidth}
  \includegraphics[width = \linewidth]{plots/geo.pdf} \\
  \begin{footnotesize}
    %% \begin{singlespace}
    %%   \emph{Notes:} 
    %% \end{singlespace}
    \end{footnotesize}
\end{minipage}
\end{figure} 

In Figure~\ref{fig:commute_vs_wfh} we plot the fraction of respondents working from home versus the fraction still commuting by US state.
There is a clear negative relationship, suggesting a fraction of current commuters will likely---or could---transition to work-from-home status.
Each 10 percentage point increase in the fraction still commuting is associated with about a 6 percentage point decline in the fraction fo workers now working from home. 

\begin{figure}
  \caption{Still commuting versus work from home fractions by US State} \label{fig:commute_vs_wfh}
\centering
\begin{minipage}{0.8 \linewidth}
  \includegraphics[width = \linewidth]{plots/commute_vs_wfh.pdf} \\
  \begin{footnotesize}
    %% \begin{singlespace}
    %%   \emph{Notes:} 
    %% \end{singlespace}
    \end{footnotesize}
\end{minipage}
\end{figure} 

A natural question is how these various measures are affecting UI claims by state. 
In Table~\ref{tab:ui}, we combine our data with the two UI claims from \cite{goldsmith2020}.
We regress the log of two weeks of UI claims by state on the state population, plus the state-specific fraction for each of the response possibilities.  
Unsurprisingly, across all specifications, the state population explains a great deal of the variance in UI claims.
Our interest is in whether and the various survey measures account for some amount of the residual variance.


\begin{table}[!htbp] \centering 
  \caption{Predicting UI claims by state} 
  \label{tab:ui} 
\small 
\begin{tabular}{@{\extracolsep{5pt}}lcccc} 
\\[-1.8ex]\hline 
\hline \\[-1.8ex] 
 & \multicolumn{4}{c}{\textit{Dependent variable:}} \\ 
\cline{2-5} 
\\[-1.8ex] & \multicolumn{4}{c}{Log state two week UI claims} \\ 
\\[-1.8ex] & (1) & (2) & (3) & (4)\\ 
\hline \\[-1.8ex] 
 Log state population & 0.967$^{***}$ & 0.936$^{***}$ & 0.996$^{***}$ & 0.992$^{***}$ \\ 
  & (0.063) & (0.066) & (0.066) & (0.068) \\ 
  Still commuting frac. (log) & $-$0.744$^{***}$ &  &  &  \\ 
  & (0.254) &  &  &  \\ 
  Switch to WFH frac. (log) &  & 0.695$^{***}$ &  &  \\ 
  &  & (0.248) &  &  \\ 
  Laid-off frac. (log) &  &  & 0.255 &  \\ 
  &  &  & (0.204) &  \\ 
  Still WFH (log) &  &  &  & 0.101 \\ 
  &  &  &  & (0.218) \\ 
  log(lfpr) & $-$0.393 & $-$0.598 & $-$0.189 & $-$0.110 \\ 
  & (0.808) & (0.825) & (0.862) & (0.878) \\ 
  Constant & $-$4.185$^{***}$ & $-$2.283$^{*}$ & $-$3.209$^{**}$ & $-$3.456$^{**}$ \\ 
  & (1.097) & (1.211) & (1.238) & (1.326) \\ 
 \hline \\[-1.8ex] 
Observations & 50 & 50 & 50 & 50 \\ 
R$^{2}$ & 0.854 & 0.852 & 0.833 & 0.828 \\ 
Adjusted R$^{2}$ & 0.845 & 0.843 & 0.822 & 0.817 \\ 
\hline 
\hline \\[-1.8ex] 
\end{tabular}
\\
\begin{minipage}{1.0 \textwidth}
{\footnotesize \emph{Notes}: 
\starlanguage}
\end{minipage}
\end{table}


In Column~(1), we append the state-specific fraction reporting that they were still commuting to work.
The higher the fraction reporting still commuting, the lower UI claims for that state.
In Column~(2), the greater the fraction that report workin from home, the \emph{higher} the UI claims.
What seems likely is that workers who would otherwise be continuing to commute to work are splitting into (a) work-from-home or (b) filing for UI.
As states reduce the fraction commuting, we should be able to get a sense of thse resulting increases UI filings. 

\section{Conclusion}
We document some early facts about how the US labor force is responding to COVID-19 pandemic.
We will continue to track changes to the nature of remote work, asking how pandemic-induced changes transform workplaces in the short and long-term.

\subsection{Suggested immediate future work} 

The code and data for this project are here: 
\begin{itemize}
\item See if state-specific occupational distributions are affecting the fractions working remotely vesus splitting in UI. 
\item See if Covid-19 deaths/hospitalizations are affecting 
\item Someone should be able to make predictions about UI claims in states that shift more of their workforce.
\end{itemize}


\newpage \clearpage
\bibliographystyle{aer}
\bibliography{remote_work.bib}


\appendix

\section{Appendix} 
\subsection{By gender} \label{sec:gender}

In Figure~\ref{fig:gender} we report responses by inferred gender.
Fractions are computed separately for males and females, and then a slope graph is used to show differences. 

\begin{figure}
  \caption{Responses by gender} \label{fig:gender}
\centering
\begin{minipage}{1.0 \linewidth}
  \includegraphics[width = \linewidth]{plots/gender.pdf} \\
  \begin{footnotesize}
    \begin{singlespace}
      \emph{Notes:} Standard errors are reported. 
    \end{singlespace}
    \end{footnotesize}
\end{minipage}
\end{figure} 

\end{document} 
